\documentclass[12pt]{article}

\usepackage[a4paper]{geometry} %page size
\usepackage{parskip} %no paragraph indentation
\usepackage{fancyhdr} %fancy stuff in page header
\pagestyle{fancy} 

\usepackage[utf8]{inputenc} %encoding
\usepackage[danish]{babel} %danish letters

\usepackage{graphicx} %import pictures

\usepackage{amsmath, amssymb, amsfonts, amsthm, mathtools} %doing math
\usepackage{algorithmicx, algpseudocode} %doing pseudocode,

\fancyhead{}
\lhead{Machine Learning: Home Assignment 3}
\rhead{Asger Andersen}

%End of preamble
%*******************************************************************************

\begin{document}

\section{To Split or Not To Split?}

\textbf{Question 1:} In this case, our hypothesis space $\mathcal{H} = \{h_1,...,h_M \}$ is finite with $|\mathcal{H}| = M$, which means that we can use \textbf{Theorem 3.2} to conclude that with probability $1 - \delta$ for all $h \in \mathcal{H}$
\begin{align}
L(\hat{h}^*) \leq \hat{L}(\hat{h}^*, S_{val}) + \sqrt{\frac{\ln \frac{M}{\delta}}{2n}}
\end{align}
where $n=|S_{val}|$.

\textbf{Question 2:} Let $S_{val}^*$ be the validation set om which we are testing the hypothesis $\hat{h}^*$ that we end up choosing. In the setup proposed by our fellow student, we are only testing a single hypothesis, namely $\hat{h}^*$, on $S_{val}^*$, and $|S_{val^*}|=\frac{n}{M}$. Therefore, we can use \textbf{Theorem 3.1} to conclude that with probability $1 - \delta$ for all $h \in \mathcal{H}$
\begin{align}
L(\hat{h}^*) \leq \hat{L}(\hat{h}^*, S_{val}) + \sqrt{\frac{\ln \frac{1}{\delta}}{2\frac{n}{M}}} = \hat{L}(\hat{h}^*, S_{val}) + \sqrt{\frac{ M \ln \frac{1}{\delta}}{2n}}
\end{align}
Our fellow student has therefore made a bad proposal, since bound is now growing linearly with $M$ instead of logarithmically.

\textbf{Question 3:} Again we only test a single hypothesis, namely $\hat{h}^*$, on $S_{val}^2$. This time we have that $|S_{val}^2|=\frac{n}{2}$. Therefore, we can use \textbf{Theorem 3.1} to conclude that with probability $1 - \delta$ for all $h \in \mathcal{H}$
\begin{align}
L(\hat{h}^*) \leq \hat{L}(\hat{h}^*, S_{val}) + \sqrt{\frac{\ln \frac{1}{\delta}}{2\frac{n}{2}}} = \hat{L}(\hat{h}^*, S_{val}) + \sqrt{\frac{ \ln \frac{1}{\delta}}{n}}
\end{align}

Assume that my fellow student followed this procedure, and I followed the procedure in question 1. Let $\hat{h}^*$ be the hypothesis that I end up choosing, and let $\tilde{h}^*$ be the hypothesis my fellow student chooses. Where I am using the full $S_{val}$ to choose $\hat{h}^*$, my fellow student is only using $S^1_{val}$ to choose $\tilde{h}^*$. Therefore, we cannot not be sure that $\hat{h}^* = \tilde{h}^*$. Apart from not knowing whether we choose the same hypothesis, we also do not test our chosen hypothesis on the same set. Where I am using $S_{val}$, my fellow student is using $S_{val}^2$. All in all, it is therefore not very easy to tell know how close my empirical error $\hat{L}(\hat{h}^*,S_{val})$ is to the empirical error $\hat{L}(\tilde{h}^*,S_{val}^2)$ of my fellow student. However, we can say that I have a higher probability of choosing the hypothesis $h_i$ in $\mathcal{H}$ with the lowest expected loss $L(h_i)$, since I am using a bigger validation set to inform my decision.

If we assume that $\hat{L}(\hat{h}^*,S_{val}) = \hat{L}(\tilde{h}^*,S_{val}^2)$, then we know that my bound is tighter than my fellow student's, if and only if
\begin{align}
\sqrt{\frac{\ln \frac{M}{\delta}}{2n}} < \sqrt{\frac{ \ln \frac{1}{\delta}}{n}} 
\end{align}
This is equivalent to
\begin{align}
\ln \frac{M}{\delta} < 2 \ln \frac{1}{\delta}
\end{align}
which is equivalent to
\begin{align}
\frac{M}{\delta} < \left( \frac{1}{\delta} \right)^2
\end{align}
which is equivalent to
\begin{align}
M \delta < 1
\end{align}
This means that under the assumption that $\hat{L}(\hat{h}^*,S_{val}) = \hat{L}(\tilde{h}^*,S_{val}^2)$, then if we for instance wanted a certainty $1-\delta = 0.95$, then my procedure would have a tighter bound, if and only if $M < 20$. 

As I had already said, then even if we had a big $M$, my fellow student would still be less certain than me of picking the best hypothesis in $\mathcal{H}$, which is a drawback of his method.

\textbf{Question 4:} As I have already explained in question 3, then choosing a large $\alpha$ - and thereby a large validation set $S_{val}^1$ - means having a better chance of choosing the hypothesis in $\mathcal{H}$, which actually has the lowest expected loss, as $\hat{h}^*$. This also means that we should expect a lower empirical loss $L(\hat{h}^*,S_{val}^2)$ on the test set $S_{val}^2$ than if we had used a smaller validation set to choose $\hat{h}^*$. However, a large $\alpha$ also means a small test set. Therefore, we also get more uncertain how well the empirical loss $L(\hat{h}^*,S_{val}^2)$ on the test set reflects the true expected loss $L(\hat{h}^*)$. This can be seen by the fact that the term
\begin{align}
\sqrt{\frac{ \ln \frac{1}{\delta}}{2(1-\alpha)n}}
\end{align}
in our bound
\begin{align}
L(\hat{h}^*) \leq = \hat{L}(\hat{h}^*, S_{val}^2) + \sqrt{\frac{ \ln \frac{1}{\delta}}{2(1-\alpha)n}}
\end{align}
grows when $\alpha$ becomes larger. All in all, it therefore not clear whether a larger $\alpha$ will make us choose $\hat{h}^*$, such that the resulting bound on $L(\hat{h}^*)$ becomes larger or smaller. In general, the larger $M$ becomes, the larger I would also choose $\alpha$, since a large hypothesis space also means a large probability of accidentally choosing a bad hypothesis as $\hat{h}^*$, if the validation set is too small.

\section{Occam's Razor}

\input{occams_razor/occams_razor.tex}

\section{Kernels}

\subsection{Distance in feature space}

\input{some_section.tex}

Let $k$ be a kernel on input space $\mathcal{X}$ defining the RKHS $\mathcal{H}$, and let $\Phi: \mathcal{X} \to \mathcal{H}$ be the corresponding feature map. Let $\langle \cdot \rangle$ be the inner product on $\mathcal{H}$, which has been defined as part of the construction of $\mathcal{H}$ as the RKHS of $k$. 

Let $x,z\in \mathcal{X}$. By definition of the canonical norm on a Hilbert space, we know that
\begin{align}
||\Phi(x) - \Phi(z) ||^2 = \langle \Phi(x) - \Phi(z),\Phi(x) - \Phi(z) \rangle
\end{align}
Since any inner product on a Hilbert space must be linear in both arguments, we get that
\begin{align}
\langle \Phi(x) - \Phi(z),\Phi(x) - \Phi(z) \rangle = 
\langle \Phi(x),\Phi(x) \rangle + \langle \Phi(z),\Phi(z) \rangle - 2\langle \Phi(x) , \Phi(z) \rangle
\end{align}
By line (19-20) we know get that
\begin{align}
||\Phi(x) - \Phi(z) || = \sqrt{\langle \Phi(x),\Phi(x) \rangle + \langle \Phi(z),\Phi(z) \rangle - 2\langle \Phi(x) , \Phi(z) \rangle}
\end{align}
By construction of the RKHS of $k$, we know that for all $x_1, x_2 \in \mathcal{X}$
\begin{align}
k(x_1,x_2)=\langle \Phi(x_1), \Phi(x_2) \rangle
\end{align}
By line (21-22) we know have that
\begin{align}
||\Phi(x) - \Phi(z) || = \sqrt{k(x,z) + k(z,z) - 2k(x,z)}
\end{align}
which is what I was asked to show.

\subsection{Sum of kernels}

Let $k_1, k_2: \mathcal{X}\times \mathcal{X} \to \mathbb{R}$ be kernels\footnote{I omit to say positive definit kernels, since it is a part of the definition of a kernel that it is positive definit.}. Let $x_1,...,x_m \in \mathcal{X}$, and let $A$ and $B$ be the Gram matrix of $k_1$ and $k_2$, respectively, with respect to $x_1,...,x_m$. Since $k_1$ and $k_2$ are kernels, $A$ and $B$ are positive definit matrices, which means
\begin{align}
\forall c_1,...,c_m \in \mathbb{R}: \sum_{i,j}^m c_i c_j A_{ij} \geq 0
\end{align}
and 
\begin{align}
\forall c_1,...,c_m \in \mathbb{R}: \sum_{i,j}^m c_i c_j B_{ij} \geq 0
\end{align}

Consider now the function $k_3:\mathcal{X}\times \mathcal{X} \to \mathbb{R}$ defined by
\begin{align}
k_3(x,y) = k_1(x,y) + k_2(x,y)
\end{align}
Let $C$ be the Gram matrix of $k_3$ with respect to $x_1,...,x_m$. By definition of $C$ and $k_3$ we have that
\begin{align}
C_{ij} = k_3(x_i,x_j) = k_1(x_i,x_j) + k_2(x_i,x_j) = A_{ij} + B_{ij}
\end{align} 
By line (24-25) and (27) we now get that
\begin{align}
\forall c_1,...,c_m \in \mathbb{R}: \sum_{i,j}^m c_i c_j C_{ij} = \sum_{i,j}^m c_i c_j (A_{ij} + B_{ij}) \\
= \sum_{i,j}^m c_i c_jA_{ij} + \sum_{i,j}^m c_i c_jB_{ij} \geq 0
\end{align}
This means that $C$ is positive definit. 

Since $x_1,...,x_m$ was arbitrary we now have that for all $m \in \mathbb{N}$ and for all $x_1,...,x_m$, then the Gram matrix of the function $k_3$ with respect to $x_1,...,x_m$ is positive definit. This means that $k_3$ is a kernel function.

All in all, I have now shown that if $k_1$ and $k_2$ are kernels on input space $\mathcal{X}$, then the function $k_3 = k_1 + k_2$ is also a kernel on $\mathcal{X}$.

\subsection{Rank of Gram matrix}

The answer is $m-d$. Like the degrees of freedom.

\end{document}