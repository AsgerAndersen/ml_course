\documentclass[12pt]{article}

\usepackage[a4paper]{geometry} %page size
\usepackage{parskip} %no paragraph indentation
\usepackage{fancyhdr} %fancy stuff in page header
\pagestyle{fancy} 

\usepackage[utf8]{inputenc} %encoding
\usepackage[danish]{babel} %danish letters

\usepackage{graphicx} %import pictures

\usepackage{amsmath, amssymb, amsfonts, amsthm, mathtools} %doing math
\usepackage{algorithmicx, algpseudocode} %doing pseudocode,

\fancyhead{}
\lhead{Machine Learning: Home Assignment 2}
\rhead{Asger Andersen}

%End of preamble
%*******************************************************************************

\begin{document}

\section{Illustration of Hoeffding's Inequality}

This is some section!

\section{The effect of scale (range) and normalization of random variables in Hoeffding's Inequality}

This is some section!

\section{Probability in Practice}

This is some section!

\section{Logistic Regression}

This is some section!

\subsection{Cross-entropy measure}

Let $\mathcal{X}$ be some sample space, and let $\mathcal{Y}$ be the label space $\{-1,1\}$, and assume that we want to learn the distribution of the labels $y$ conditioned on the value of a sample $x$, that is we want to learn the conditional probability $P(y|x)$ for $y \in {-1,1}$ and all $x\in \mathcal{X}$. Also, assume that the distribution $P(y|x)$ can be parametrized by choosing $w$ in some parameter space $\mathcal{W}$. That is, by choosing $w \in \mathcal{W}$ we get the value of $P_w(y|x)$ for $y \in {-1,1}$ and all $x\in \mathcal{X}$. In this context, the learning problem becomes to come up with a method for choosing some parameter $\hat{w} \in \mathcal{W}$ and hereby a corresponding distribution $P_{\hat{w}}(y|x)$, which somehow is our best guess of the true distribution of $y$ conditioned on $x$. The information we have available to base this choice on is some finite, labeled sample $S = \{(x_1,y_1),...,(x_N,y_N)\}$, where each $y_i$ is assumed to have been sampled from $P_w(y|x_i)$ and all of them independently from each other.

The maximum likelihood method for choosing $\hat{w}$ solves this problem by defining the likelihood function $L_S$ for the given sample $S$ as
\begin{align}
L_S(w) = \prod_{(x_n,y_n)\in \mathcal{S}}  P_w(y_n|x_n)
= \prod_{n=1}^N P_w(y_n|x_n)
\end{align}
and then saying that we should choose $\hat{w} \in \mathcal{W}$ such that $L_\mathcal{S}$ is maximized.

Since the function $-\ln$ is monotonically decreasing, this strategy is equivalent\footnote{I define two strategies to be equivalent, if and only if they end up choosing the same $\hat{w}$ for all possible samples $S = \{(x_1,y_1),...,(x_N,y_N)\}$.} to choosing $\hat{w} \in \mathcal{W}$ such that the function
\begin{align}
f_S(w) = -\ln \left( \prod_{n=1}^N P_w(y_n|x_n) \right) = \sum_{n=1}^N \left( - \ln P_w(y_n|x_n) \right)
\end{align}
is minimized. 

Since $y \in \{-1,1\}$, then we can write $P_w(y|x)$ as
\begin{align}
P_w(y|x)  = \\ 
[[y = 1]] P_w(y = 1|x) + [[y = 1]] P_w(y = -1|x) = \\ 
[[y = 1]] h_w(x) + [[y = 1]] (1 - h_w(x))
\end{align}
where we simply have defined $h_w(x) = P(y = 1 | x)$. We therefore have that
\begin{align}
- \ln P_w(y_n|x_n) = \\ 
-\ln ([[y = 1]] h_w(x) + [[y = -1]] (1 - h_w(x))) = \\ 
[[y = 1]](-\ln (h_w(x)) + [[y = -1]](-\ln (1 - h_w(x))) = \\ 
[[y = 1]]\left(\ln \left( \frac{1}{h_w(x)}\right)\right) + [[y = -1]]\left(\ln \left( \frac{1}{1 - h_w(x)}\right)\right)
\end{align}

By line (2) and line (6-9), we now get that
\begin{align}
f_S(w) = \sum_{n=1}^N \left[ [[y_n = 1]]\left(\ln \left( \frac{1}{h_w(x_n)}\right)\right) + [[y_n = -1]]\left(\ln \left( \frac{1}{1 - h_w(x_n)}\right)\right) \right]
\end{align}
As I have already said, we will end up with the same $\hat{w}$ for a given sample $S$, if we minimize $f_S(w)$ as if we maximize $L_S(w)$. If we had started by saying that we would like to estimate the probability $h_w(x) = P_w(y = 1|x)$ by choosing $\hat{w}$ such that we minimize the error function $f_S(w)$ defined as in line (10), we would have ended up with the same estimates of $P(y|x)$ for $y \in \{-1,1\}$ and $x\in \mathcal{X}$, as if we have used the maximum likelihood method. These two strategies are therefore equivalent.

\end{document}