\documentclass[12pt]{article}

\usepackage[a4paper]{geometry} %page size
\usepackage{parskip} %no paragraph indentation
\usepackage{fancyhdr} %fancy stuff in page header
\pagestyle{fancy} 

\usepackage[utf8]{inputenc} %encoding
\usepackage[danish]{babel} %danish letters

\usepackage{graphicx} %import pictures

\usepackage{amsmath, amssymb, amsfonts, amsthm, mathtools} %doing math
\usepackage{algorithmicx, algpseudocode} %doing pseudocode,

\fancyhead{}
\lhead{Machine Learning: Home Assignment 2}
\rhead{Asger Andersen}

%End of preamble
%*******************************************************************************

\begin{document}

\section{Illustration of Hoeffding's Inequality}

This is some section!

\section{The effect of scale (range) and normalization of random variables in Hoeffding's Inequality}

Let all the assumptions of \textbf{corollary 2.4} be true for some random variables $X_1, ..., X_n$. Set $a_i = 0$ and $b_i=1$ for all $i \in \{1, ..., n\}$. By the assumptions of \textbf{corollary 2.4} and our definition of $a_i$ and $b_i$, we now have that all the assumptions of \textbf{theorem 2.2} are true for $X_1, ..., X_n$. We can therefore use \textbf{theorem 2.2} to conclude that for all $\varepsilon >0$
\begin{align}
\mathbb{P}\left\{ \sum_{i=1}^n X_i - \mathbb{E}\left[ \sum_{i=1}^n X_i \right] \geq \varepsilon \right\} \leq e^{-2 \varepsilon^2 / \sum_{i=1}^n (b_i - a_i)^2}
\end{align}
and
\begin{align}
\mathbb{P}\left\{ \sum_{i=1}^n X_i - \mathbb{E}\left[ \sum_{i=1}^n X_i \right] \leq - \varepsilon \right\} \leq e^{-2 \varepsilon^2 / \sum_{i=1}^n (b_i - a_i)^2}
\end{align}

Since the assumptions of the corollary states that $\mathbb{E}(X_i) = \mu$ for all $i \in \{1, ..., n\}$, then we know that 
\begin{align}
\mathbb{E}\left[ \sum_{i=1}^n X_i \right] = n\mu 
\end{align}
Since $b_i - a_i = 1$ for all $i \in \{1, ..., n\}$, we also know that
\begin{align}
\sum_{i=1}^n (b_i - a_i)^2 = n 
\end{align}
From line (XX-XX) we therefore get that for all $\varepsilon > 0$
\begin{align}
\mathbb{P}\left\{ \sum_{i=1}^n X_i - n\mu \geq \varepsilon \right\} \leq e^{- 2 \frac{\varepsilon^2}{n} } = e^{- 2n \left( \frac{ \varepsilon}{n} \right)^2 }
\end{align}
and
\begin{align}
\mathbb{P}\left\{ \sum_{i=1}^n X_i - n\mu \leq -\varepsilon \right\} \leq e^{- 2 \frac{\varepsilon^2}{n} } = e^{- 2n \left( \frac{ \varepsilon}{n} \right)^2 }
\end{align}
From this it clearly follows that for all $\varepsilon > 0$
\begin{align}
\mathbb{P}\left\{\frac{1}{n} \sum_{i=1}^n X_i - \mu \geq \frac{\varepsilon}{n} \right\} \leq = e^{- 2n \left( \frac{ \varepsilon}{n} \right)^2 }
\end{align}
and
\begin{align}
\mathbb{P}\left\{\frac{1}{n} \sum_{i=1}^n X_i - \mu \leq- \frac{\varepsilon}{n} \right\} \leq = e^{- 2n \left( \frac{ \varepsilon}{n} \right)^2 }
\end{align}
If $\varepsilon > 0$, then $\tilde{\varepsilon} = \frac{\varepsilon}{n} > 0$ for all $n\in \mathbb{N}$. We can therefore now conclude that for all $\tilde{\varepsilon}>0$
\begin{align}
\mathbb{P}\left\{\frac{1}{n} \sum_{i=1}^n X_i - \mu \geq \tilde{\varepsilon} \right\} \leq = e^{- 2n \tilde{\varepsilon}^2 }
\end{align}
and
\begin{align}
\mathbb{P}\left\{\frac{1}{n} \sum_{i=1}^n X_i - \mu \leq - \tilde{\varepsilon} \right\} \leq = e^{- 2n \tilde{\varepsilon}^2 }
\end{align}
We have now proven that \textbf{corollary 2.4} follows from \textbf{theorem 2.2}.


\section{Probability in Practice}

This is some section!

\section{Logistic Regression}

This is some section!

\subsection{Cross-entropy measure}

Let $\mathcal{X}$ be some sample space, and let $\mathcal{Y}$ be the label space $\{-1,1\}$, and assume that we want to learn the distribution of the labels $y$ conditioned on the value of a sample $x$, that is we want to learn the conditional probability $P(y|x)$ for $y \in {-1,1}$ and all $x\in \mathcal{X}$. Also, assume that the distribution $P(y|x)$ can be parametrized by choosing $w$ in some parameter space $\mathcal{W}$. That is, by choosing $w \in \mathcal{W}$ we get the value of $P_w(y|x)$ for $y \in {-1,1}$ and all $x\in \mathcal{X}$. In this context, the learning problem becomes to come up with a method for choosing some parameter $\hat{w} \in \mathcal{W}$ and hereby a corresponding distribution $P_{\hat{w}}(y|x)$, which somehow is our best guess of the true distribution of $y$ conditioned on $x$. The information we have available to base this choice on is some finite, labeled sample $S = \{(x_1,y_1),...,(x_N,y_N)\}$, where each $y_i$ is assumed to have been sampled from $P_w(y|x_i)$ and all of them independently from each other.

The maximum likelihood method for choosing $\hat{w}$ solves this problem by defining the likelihood function $L_S$ for the given sample $S$ as
\begin{align}
L_S(w) = \prod_{(x_n,y_n)\in \mathcal{S}}  P_w(y_n|x_n)
= \prod_{n=1}^N P_w(y_n|x_n)
\end{align}
and then saying that we should choose $\hat{w} \in \mathcal{W}$ such that $L_\mathcal{S}$ is maximized.

Since the function $-\ln$ is monotonically decreasing, this strategy is equivalent\footnote{I define two strategies to be equivalent, if and only if they end up choosing the same $\hat{w}$ for all possible samples $S = \{(x_1,y_1),...,(x_N,y_N)\}$.} to choosing $\hat{w} \in \mathcal{W}$ such that the function
\begin{align}
f_S(w) = -\ln \left( \prod_{n=1}^N P_w(y_n|x_n) \right) = \sum_{n=1}^N \left( - \ln P_w(y_n|x_n) \right)
\end{align}
is minimized. 

Since $y \in \{-1,1\}$, then we can write $P_w(y|x)$ as
\begin{align}
P_w(y|x)  = \\ 
[[y = 1]] P_w(y = 1|x) + [[y = 1]] P_w(y = -1|x) = \\ 
[[y = 1]] h_w(x) + [[y = 1]] (1 - h_w(x))
\end{align}
where we simply have defined $h_w(x) = P(y = 1 | x)$. We therefore have that
\begin{align}
- \ln P_w(y_n|x_n) = \\ 
-\ln ([[y = 1]] h_w(x) + [[y = -1]] (1 - h_w(x))) = \\ 
[[y = 1]](-\ln (h_w(x)) + [[y = -1]](-\ln (1 - h_w(x))) = \\ 
[[y = 1]]\left(\ln \left( \frac{1}{h_w(x)}\right)\right) + [[y = -1]]\left(\ln \left( \frac{1}{1 - h_w(x)}\right)\right)
\end{align}

By line (2) and line (6-9), we now get that
\begin{align}
f_S(w) = \sum_{n=1}^N \left[ [[y_n = 1]]\left(\ln \left( \frac{1}{h_w(x_n)}\right)\right) + [[y_n = -1]]\left(\ln \left( \frac{1}{1 - h_w(x_n)}\right)\right) \right]
\end{align}
As I have already said, we will end up with the same $\hat{w}$ for a given sample $S$, if we minimize $f_S(w)$ as if we maximize $L_S(w)$. If we had started by saying that we would like to estimate the probability $h_w(x) = P_w(y = 1|x)$ by choosing $\hat{w}$ such that we minimize the error function $f_S(w)$ defined as in line (10), we would have ended up with the same estimates of $P(y|x)$ for $y \in \{-1,1\}$ and $x\in \mathcal{X}$, as if we have used the maximum likelihood method. These two strategies are therefore equivalent.

\subsection{Logistic regression loss gradient}

In the algorithm for logistic regression, we determine the parameter $w \in \mathbb{R}^m$ in our final hypothesis
\begin{align}
h_w(x) = P_w(y = 1|x) = \frac{e^{w^Tx}}{1 + e^{w^Tx}}
\end{align}
using a given labeled sample $S = \{(x_1,y_1),...,(x_N,y_N)\}$ by minizing the function
\begin{align}
E_{in}(w) = \frac{1}{N} \sum_{n=1}^N \ln(1 + e^{-y_n w^T x_n})
\end{align}
We use gradient descent to do this, which means we have to find the loss gradient $\nabla E_{in}(w)$. We can use matrix calculus to to differentiate $E_{in}$
\begin{align}
D_w (E_{in}(w)) = D_w\left( \frac{1}{N} \sum_{n=1}^N \ln(1 + e^{-y_n w^T x_n}) \right) = \\ 
 \frac{1}{N} \sum_{n=1}^N D_w \left( \ln(1 + e^{-y_n w^T x_n}) \right). 
\end{align}
Let us now focus on the inside of the sum
\begin{align}
D_w \left( \ln(1 + e^{-y_n w^T x_n}) \right) = \frac{D_w ( 1 + e^{-y_n w^T x_n})}{1 + e^{-y_n w^T x_n}} = \\ 
\frac{D_w (e^{-y_n w^T x_n})}{1 + e^{-y_n w^T x_n}} =
\frac{e^{-y_n w^T x_n} D_w(-y_nw^Tx)}{1 + e^{-y_n w^T x_n}} = \\ 
-y_n x_n \frac{e^{-y_n w^T x_n}}{1 + e^{-y_n w^T x_n}} = -y_n x_n \Theta(-y_n w^T x_n)
\end{align}
Line (13-14) and line (15-17) now gives us that
\begin{align}
D_w (E_{in}(w)) =  \frac{1}{N} \sum_{n=1}^N D_w \left( \ln(1 + e^{-y_n w^T x_n}) \right) =
\frac{1}{N} \sum_{n=1}^N  -y_n x_n \Theta(-y_n w^T x_n) 
\end{align}
which is what the exercise asked us to show. 

\end{document}